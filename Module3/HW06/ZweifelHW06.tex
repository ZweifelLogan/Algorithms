% ===============================================================
%
%  Template for creating scribe notes for CS:3330, Algorithms.             I am using this template to get my homework PDF's set up as well
%
%  Fill in your name, lecture date, and body of scribe notes
%  as indicated below.
%
% ===============================================================

\documentclass[11pt]{article}

\usepackage{graphicx}
\usepackage{amssymb, amsthm}
\usepackage{pgfplots}
\usepackage{tikz}
\usetikzlibrary{datavisualization}
\usetikzlibrary{datavisualization.formats.functions}
\usepackage{mathtools}
\usepackage{amsmath}
\usepackage{algorithmicx}
\usepackage{algorithm}
\usepackage{algpseudocode}



\setlength{\topmargin}{0pt}
\setlength{\textheight}{9in}
\setlength{\headheight}{0pt}
\setlength{\headsep}{0pt}
\setlength{\oddsidemargin}{0.25in}
\setlength{\textwidth}{6in}

\pagestyle{plain}

\begin{document}

\thispagestyle{empty}

\begin{center}
\bf\large CS:3330, Algorithms
\end{center}

\begin{center}
\bf\large HW06 Recurrences  %Fill in Name of Homework here
\end{center}

\noindent
Logan Zweifel     % FILL IN YOUR NAME HERE
\hfill
September 26, 2021           % FILL IN HW DATE HERE

\noindent
\rule{\textwidth}{1pt}

\medskip

%%%%%%%%%%%%%%%%%%%%%%%%%%%%%%%%%%%%%%%%%%%%%%%%%%%%%%%%%%%%%%%%
% BODY OF HOMEWORK NOTES GOES HERE
%%%%%%%%%%%%%%%%%%%%%%%%%%%%%%%%%%%%%%%%%%%%%%%%%%%%%%%%%%%%%%%%

\section{$A_k$} %%%%%%%%%%%%%%%%%%%%%%%%%%%%%%%%%%%                        1
Consider the following recurrence:

\[ a_k =
  \begin{cases}
    0       & \quad \text{if }  k \text{=0}\\
    a_{k-1}+3k+1       & \quad \text{if } k \text{$>0$}
  \end{cases}
\]

\subsection*{a) Write out the first six terms of the recurrence.}

\bigskip
\bigskip

\begin{align*}
a_0 &= 0 = 0 = 0 \\
a_1 &= a_0 + 3(1) +1 = 0 + 3 + 1 = 4 \\
a_2 &= a_1 + 3(2) +1 = 4 + 6 + 1 = 11 \\
a_3 &= a_2 + 3(3) +1 = 11 + 9 + 1 = 21 \\
a_4 &= a_3 + 3(4) +1 = 21 + 12 + 1 = 34 \\
a_5 &= a_4 + 3(5) +1 = 34 + 15 + 1 = 50 \\
a_6 &= a_5 + 3(6) +1 = 50 + 18 + 1 = 69 
\end{align*}

\bigskip


\subsection*{b) Make a guess for the explicit formula for $a_k$.}

\bigskip
\bigskip
The first pattern that I see in the first terms denoted in part a of the problem is that there is a summation from 1 to k taking place, $[1+2+3+\dots+k]$. I saw this because every $a_{k-1}$ term can be expressed as the sum of all the terms lower than it, thus there is a summation from 1 to k. Per example A.1 in Appendix A of the textbook,

\begin{equation*}
1+2+ \dots + k = \frac{k(k+1)}{2}
\end{equation*}

\bigskip
In addition, this summation is multiplied by 3 on every term. Using this information, the $a_{k-1} + 3k$ portion of the recurrence is taken care of. The $+1$ in every term of the recurrence can just be represented as $k$ as adding $1$ to itself $k$ times will just equal $k$. Therefore the first half of the explicit formula is,

\begin{equation*}
3[1+2+3+\dots+k]+k
\end{equation*}

\noindent The following is the work for simplifying the first half of the explicit equation using substitution for the summation from $1$ to $k$ stated earlier.

\begin{eqnarray*}
3[1+2+3+\dots+k] +k &=& 3 \frac{k(k+1)}{2}+k \\
	&=& \frac{3k(k+1)}{2}+k \\
	&=& \frac{3k^2 +3k}{2}+k \\
	&=& \frac{3k^2 +3k}{2} + \frac{2k}{2} \\
	&=& \frac{3k^2 +5k}{2}
\end{eqnarray*}

\noindent This gives the final explicit formula for $a_k$ as,

\begin{equation*}
3[1+2+3+\dots+k] +k = \frac{3k^2 +5k}{2}
\end{equation*}

\bigskip


\subsection*{c) Prove your guess is correct using induction.}

\bigskip

\noindent I show that for all positive integers k, that 

\begin{equation*}
3[1+2+3+\dots+k] +k = \frac{3k^2 +5k}{2}
\end{equation*}

\bigskip

\noindent Induction base: For n=1,

\begin{eqnarray*}
3[1]+1 &=& \frac{3(1)^2 + 5(1)}{2} \\
4 &=& \frac{3+5}{2} \\
4 &=& 4
\end{eqnarray*}

\bigskip

\noindent Induction Hypothesis: Assume, for an arbitrary positive integer n, that 

\begin{equation*}
3[1+2+3+\dots+k] +k = \frac{3k^2 +5k}{2}
\end{equation*}

\bigskip

\noindent Induction step: Must show that

\begin{equation*}
3[1+2+3+\dots+(k+1)] +(k+1) = \frac{3(k+1)^2 +5(k+1)}{2}
\end{equation*}

\noindent First, I am going to simplify the right side of this equation.

\begin{eqnarray*}
\frac{3(k+1)^2 +5(k+1)}{2} &=& \frac{3k^2+6k+3+5k+5}{2} \\
	&=& \frac{3k^2+5k+6k+8}{2} \\
	&=& \frac{3k^2+5k}{2} + \frac{6k+8}{2} \\
	&=& \frac{3k^2+5k}{2} + 3k+4
\end{eqnarray*}

\noindent Now, I will show that the left side of the of the induction hypothesis for $k+1$ is equal to the simplified right side.

\begin{eqnarray*}
3[1+2+3+\dots+(k+1)] +(k+1) &=& 3[1+2+\dots+k] + k + 3(k+1)+1 \\
	&=& \frac{3k^2+5k}{2} + 3(k+1) +1 \\
	&=& \frac{3k^2+5k}{2} + 3k +3+1\\
	&=& \frac{3k^2+5k}{2} +3k+4
\end{eqnarray*}

\noindent Inbetween the first and second equations above, the induction hypothesis was substituted into the equation. The induction hypothesis is true for $k+1$ and therefore it is has been proven via induction.


\section{Master Theorem} %%%%%%%%%%%%%%%%%%%%%%%%%%%%%%                               2
Use the Master Theorem to solve the recurrence

\begin{equation*}
W(n) = 4W(\frac{n}{2}) +n
\end{equation*}


%%% INSERT DEFINITION OF MASTER THEOREM HERE
\bigskip
\bigskip

The following is the definition of the Master Theorem. If $f(n) \in \Theta (n^d)$, then

\[ T(n) =
  \begin{cases}
    \Theta (n^d)       & \quad \text{if }   \text{$a < b^d$}\\
    \Theta (n^d \log n)    & \quad \text{if }  \text{$a = b^d$}\\
    \Theta (n^{\log_b a})  & \quad \text{if }  \text{$a > b^d$}
  \end{cases}
\]


\bigskip
\bigskip
\bigskip



For this problem, following the defintion of the Master Theorem, $a=4$, $b=2$, $f(n)=n$ and $d=1$ as that is the degree of $f(n)$. $b^d \equiv 2^1=2$. In this situation, the third case of the Master theorem applies as $a>b^d \equiv 4>2$. Therefore, 

\begin{equation*}
W(n) \in \Theta (n^{\log_b a}) \equiv W(n) \in \Theta (n^2)
\end{equation*}

\noindent because $\log_b a \equiv log_4 2 = 2$.



\section{Find Largest Index}%%%%%%%%%%%%%%%%%%%%%%%%%%%%%                         3


\subsection*{a) Write the pseudocode for a divide-and-conquer algorithm that finds an index for the largest element in a list of n numbers.}

\bigskip
\bigskip


\begin{algorithmic}

\Function{findLargest}{$S, low, high$} \Comment{initial parameters are (S, 0, n)}
\If{$right - left == 1$}
	\State \Return left 
\EndIf
\State $middle = \lfloor (low + high)/2 \rfloor$
\State $maxLeft = \Call{findLargest}{$S, low, middle$}$
\State $maxRight = \Call{findLargest}{$S, middle, high$}$
\If{$S[maxLeft] \geq S[maxRight]$}
	\State \Return $maxLeft$
\Else
	\State \Return $maxRight$
\EndIf
\EndFunction

\end{algorithmic}


\subsection*{b) What will be your algorithms output for lists with several elements of largest value}


\bigskip
In the event there are multiple largest values in the list, the index returned will be of the leftmost/earliest in the list largest value.


\subsection*{c) Set up a recurrence relation for the number of key comparisons made by your algorithm}

\bigskip
\bigskip
The recurrence relation for the algorithm written in part a of this problem is:

\begin{eqnarray*}
T(n) &=& 2*T(\frac{n}{2}) +1 \\
T(1) &=& 0
\end{eqnarray*}

\bigskip

The values for this recurrence relation were identified because for every recursion, there are 2 smaller instances created with the size $\frac{n}{2}$, and the $+1$ was identified because there is only 1 simple comparison needed between splitting a larger instance to a smaller one and combining the solutions of smaller instances with each other. The initial condition was identified as $T(1)=0$ as when n=1 the base/termination case is triggered before any recursion or comparisons to key elements are executed.


\subsection*{d) Solve the recurrence relation set up in the previous part.}

\bigskip
\bigskip
Similar to problem 2, and thus following the definition stated in it, I will be using the master theorem to solve the recurrence relation from part c. Per the definition of the Master Theorem, $a=2$, $b=2$, $f(n)=1$ and $d=0$ as that is the degree of $f(n)$. $b^d \equiv 2^0=1$. In this situation, the third case of the Master theorem applies as $a>b^d \equiv 2>1$. Therefore, 

\begin{equation*}
T(n) \in \Theta (n^{\log_b a}) \equiv T(n) \in \Theta (n)
\end{equation*}

\noindent because $\log_b a \equiv log_2 2 = 1$.




%%%%%%%%%%%%%%%%%%%%%%%%%%%%%%%%%%%%%%%%%%%%%%%%%%%%%%%%%%%%%%%%

\end{document}