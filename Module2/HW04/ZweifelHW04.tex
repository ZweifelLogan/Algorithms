% ===============================================================
%
%  Template for creating scribe notes for CS:3330, Algorithms.             I am using this template to get my homework PDF's set up as well
%
%  Fill in your name, lecture date, and body of scribe notes
%  as indicated below.
%
% ===============================================================

\documentclass[11pt]{article}

\usepackage{graphicx}
\usepackage{amssymb, amsthm}
\usepackage{pgfplots}
\usepackage{tikz}
\usetikzlibrary{datavisualization}
\usetikzlibrary{datavisualization.formats.functions}
\usepackage{mathtools}
\usepackage{amsmath}



\setlength{\topmargin}{0pt}
\setlength{\textheight}{9in}
\setlength{\headheight}{0pt}
\setlength{\headsep}{0pt}
\setlength{\oddsidemargin}{0.25in}
\setlength{\textwidth}{6in}

\pagestyle{plain}

\begin{document}

\thispagestyle{empty}

\begin{center}
\bf\large CS:3330, Algorithms
\end{center}

\begin{center}
\bf\large HW04 Order Problems    %Fill in Name of Homework here
\end{center}

\noindent
Logan Zweifel     % FILL IN YOUR NAME HERE
\hfill
September 12, 2021           % FILL IN HW DATE HERE

\noindent
\rule{\textwidth}{1pt}

\medskip

%%%%%%%%%%%%%%%%%%%%%%%%%%%%%%%%%%%%%%%%%%%%%%%%%%%%%%%%%%%%%%%%
% BODY OF HOMEWORK NOTES GOES HERE
%%%%%%%%%%%%%%%%%%%%%%%%%%%%%%%%%%%%%%%%%%%%%%%%%%%%%%%%%%%%%%%%


%%%%%%%%%%1   Big-O
\section{Big-O \#1}
Show directly, using the definition of Big-O, that $2n^2+9n \in O(n^2)$.


\bigskip
\bigskip
The following is the definition for Big-O which will be used for problems 1-3 on this assignment. For a given complexity function $f(n)$, $O(f(n))$ is the set of complexity functions $g(n)$ for which there exists some positive real constant $c$ and some nonnegative integer $N$ such that for all $n \geq N$,

\begin{equation*}
g(n) \leq c * f(n)
\end{equation*}

\bigskip
\bigskip

\noindent
I show that $2n^2+9n \in O(n^2)$. Because, for $n \geq 1$,
\begin{eqnarray*}
2n^2 + 9n &\leq& 2n^2 + 9n^2 \\
&\leq& 11n^2
\end{eqnarray*}
Where $c=11$ and $N=1$ were used to obtain the result.



%%%%
\section{Big-O \#2}
Show directly, using the definition of Big-O, that $5n^2+10 \in O(n^3)$.

\bigskip
\bigskip

\noindent
I show that $5n^2+10 \in O(n^2)$. Because, for $n \geq 2$,
\begin{equation*}
5n^2 + 10 \leq 5n^3
\end{equation*}
Where $c=5$ and $N=2$ were used to obtain the result.


%%%%
\section{Big-O \#3}
Show directly, using the definition of Big-O, that $6n^2+12n \in O(n^2)$.


\bigskip
\bigskip
\noindent
I show that $6n^2+12n \in O(n^2)$. Because, for $n \geq 1$,
\begin{eqnarray*}
6n^2 + 12n &\leq& 6n^2 + 12n^2 \\
&\leq& 18n^2
\end{eqnarray*}
Where $c=18$ and $N=1$ were used to obtain the result.

%%%%%%%%%%%2  Omega
\section{Omega \# 1}
Show directly, using the definition of $\Omega$, that $6n^3-12n \in \Omega(n^3)$.


\bigskip
\bigskip
The following is the definition for Omega and will be used for problems 4-6 on this assignment. For a given complexity function $f(n)$, $\Omega(f(n))$ is the set of complexity functions $g(n)$ for which there exists some positive real constant $c$ and some nonnegative integer $N$ such that for all $n \geq N$,

\begin{equation*}
g(n) \geq c * f(n)
\end{equation*}

\bigskip
\bigskip
\noindent
I show that $6n^3-12n \in \Omega(n^3)$. Because, for $n \geq 2$,
\begin{equation*}
6n^3-12n \geq 1*n^3
\end{equation*}
Where $c=1$ and $N=2$ were used to obtain the result. For this question and the other two $\Omega$ questions on this HW, the constant was picked as 1 before the N, because that keeps the right side of the inequality as low as possible. Then, the N value was calculated by doing a simple calculation as to when N is the lowest and the inequality is true.

%%%%
\section{Omega \# 2}
Show directly, using the definition of $\Omega$, that $4n^3+2n^2 \in \Omega(n^2)$.

\bigskip
\bigskip
\noindent
I show that $4n^3+2n^2 \in \Omega(n^2)$. Because, for $n \geq 0$,
\begin{equation*}
4n^3+2n^2 \geq 1*n^2
\end{equation*}
Where $c=1$ and $N=0$ were used to obtain the result.


%%%%
\section{Omega \# 3}
Show directly, using the definition of $\Omega$, that $6n^2+12n \in \Omega(n^2)$.


\bigskip
\bigskip
\noindent
I show that $6n^2+12n \in \Omega(n^2)$. Because, for $n \geq 0$,
\begin{equation*}
6n^2+12n \geq 1*n^2
\end{equation*}
Where $c=1$ and $N=0$ were used to obtain the result.












%%%%%%%%%%%%%%%%%%%%%%%%%%%%%%%%%%%%%%%%%%%%%%%%%%%%%%%%%%%%%%%%

\end{document}
